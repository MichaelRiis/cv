%%%%%%%%%%%%%%%%%%%%%%%%%%%%%%%%%%%%%%%%%%%%%%%%%%%%%%%%%%%%%%%%%%%%%%%%
%%%%%%%%%%%%%%%%%%%%%% Simple LaTeX CV Template %%%%%%%%%%%%%%%%%%%%%%%%
%%%%%%%%%%%%%%%%%%%%%%%%%%%%%%%%%%%%%%%%%%%%%%%%%%%%%%%%%%%%%%%%%%%%%%%%

%%%%%%%%%%%%%%%%%%%%%%%%%%%%%%%%%%%%%%%%%%%%%%%%%%%%%%%%%%%%%%%%%%%%%%%%
%% NOTE: If you find that it says                                     %%
%%                                                                    %%
%%                           1 of ??                                  %%
%%                                                                    %%
%% at the bottom of your first page, this means that the AUX file     %%
%% was not available when you ran LaTeX on this source. Simply RERUN  %%
%% LaTeX to get the ``??'' replaced with the number of the last page  %%
%% of the document. The AUX file will be generated on the first run   %%
%% of LaTeX and used on the second run to fill in all of the          %%
%% references.                                                        %%
%%%%%%%%%%%%%%%%%%%%%%%%%%%%%%%%%%%%%%%%%%%%%%%%%%%%%%%%%%%%%%%%%%%%%%%%
%%%%%%%%%%%%%%%%%%%%%%%%%%%% Document Setup %%%%%%%%%%%%%%%%%%%%%%%%%%%%

% Don't like 10pt? Try 11pt or 12pt
\documentclass[10pt]{article}
\RequirePackage[T1]{fontenc}

% The automated optical recognition software used to digitize resume
% information works best with fonts that do not have serifs. This
% command uses a sans serif font throughout. Uncomment both lines (or at
% least the second) to restore a Roman font (i.e., a font with serifs).
\usepackage{times}
\renewcommand{\familydefault}{\sfdefault}

% The OCR software also has a hard time with italics. These commands get
% rid of the two common ways to italicize text in LaTeX. Get rid of them
% to turn italics back on.
\renewcommand\emph[1]{#1}
%\renewcommand\textit[1]{\underline{#1}}

% This is a helpful package that puts math inside length specifications
\usepackage{calc}

% This package helps LaTeX auto-hyphenate hyphenated words if you use
% special hyphens. For example, bio\-/mimicry will properly hyphenate
% ``mimicry'' if necessary.
\usepackage[shortcuts]{extdash}

% Layout: Puts the section titles on left side of page
\reversemarginpar

%
%         PAPER SIZE, PAGE NUMBER, AND DOCUMENT LAYOUT NOTES:
%
% The next \usepackage line changes the layout for CV style section
% headings as marginal notes. It also sets up the paper size as either
% letter or A4. By default, letter was used. If A4 paper is desired,
% comment out the letterpaper lines and uncomment the a4paper lines.
%
% As you can see, the margin widths and section title widths can be
% easily adjusted.
%
% ALSO: Notice that the includefoot option can be commented OUT in order
% to put the PAGE NUMBER *IN* the bottom margin. This will make the
% effective text area larger.
%
% IF YOU WISH TO REMOVE THE ``of LASTPAGE'' next to each page number,
% see the note about the +LP and -LP lines below. Comment out the +LP
% and uncomment the -LP.
%
% IF YOU WISH TO REMOVE PAGE NUMBERS, be sure that the includefoot line
% is uncommented and ALSO uncomment the \pagestyle{empty} a few lines
% below.
%

%% Use these lines for letter-sized paper
\usepackage[paper=letterpaper,
            %includefoot, % Uncomment to put page number above margin
            marginparwidth=1.2in,     % Length of section titles
            marginparsep=.05in,       % Space between titles and text
            margin=1in,               % 1 inch margins
            includemp]{geometry}

%% Use these lines for A4-sized paper
%\usepackage[paper=a4paper,
%            %includefoot, % Uncomment to put page number above margin
%            marginparwidth=30.5mm,    % Length of section titles
%            marginparsep=1.5mm,       % Space between titles and text
%            margin=25mm,              % 25mm margins
%            includemp]{geometry}

%% More layout: Get rid of indenting throughout entire document
\setlength{\parindent}{0in}

% Provides special list environments and macros to create new ones
\usepackage[shortlabels]{enumitem}

% Simpler bibsections for CV sections
% (thanks to natbib for inspiration)
%
% * For lists of references with hanging indents and no numbers:
%
%   \begin{bibsection}
%       \item ...
%   \end{bibsection}
%
% * For numbered lists of references (with hanging indents):
%
%   \begin{bibenum}
%       \item ...
%   \end{bibenum}
%
%   Note that bibenum numbers continuously throughout. To reset the
%   counter, use
%
%   \restartlist{bibenum}
%
%   at the place where you want the numbering to reset.

\makeatletter
\newlength{\bibhang}
\setlength{\bibhang}{1em}
\newlength{\bibsep}
 {\@listi \global\bibsep\itemsep \global\advance\bibsep by\parsep}
\newlist{bibsection}{itemize}{3}
\setlist[bibsection]{label=,leftmargin=\bibhang,%
        itemindent=-\bibhang,
        itemsep=\bibsep,parsep=\z@,partopsep=0pt,
        topsep=0pt}
\newlist{bibenum}{enumerate}{3}
\setlist[bibenum]{label=[\arabic*],resume,leftmargin={\bibhang+\widthof{[999]}},%
        itemindent=-\bibhang,
        itemsep=\bibsep,parsep=\z@,partopsep=0pt,
        topsep=0pt}
\let\oldendbibenum\endbibenum
\def\endbibenum{\oldendbibenum\vspace{-.6\baselineskip}}
\let\oldendbibsection\endbibsection
\def\endbibsection{\oldendbibsection\vspace{-.6\baselineskip}}
\makeatother

%%% Setup header and footer (with page number and possible last page)
%
% The first block sets up pages 2--end
% The second block sets up page 1 formatting
%
%%%
%
% NOTE: comment the +LP lines and uncomment the -LP lines to have page
%       numbers without the ``of ##'' last page reference)
%
% NOTE: uncomment the \pagestyle{empty} line to get rid of all page
%       numbers on pages 2--end. To get rid of page numbers on page 1,
%       comment out the \thispagestyle{plain} line on the first page
%       below.
%       (also make sure includefoot is commented out above)
%
\usepackage{fancyhdr,lastpage}
\pagestyle{fancy}
%\pagestyle{empty}      % Uncomment this to get rid of page numbers
\fancyhf{}\renewcommand{\headrulewidth}{0pt}
\fancyfootoffset{\marginparsep+\marginparwidth}
\newlength{\footpageshift}
\setlength{\footpageshift}
          {0.5\textwidth+0.5\marginparsep+0.5\marginparwidth-2in}

%%%% PAGES 2--9 NUMBERING:
%% These two lines put page number in upper-right corner of pages 2--end
\rhead{Michael Riis Andersen, CV \& Publication list, p.~\arabic{page} of \protect\pageref*{LastPage}}   % +LP
%\rhead{Pavlic, p.~\arabic{page}}                                 % -LP

%% These lines put page number in bottom (center) of pages 2--end
%\lfoot{\hspace{\footpageshift}%
%       \parbox{4in}{\, \hfill %
%                    \arabic{page} of \protect\pageref*{LastPage} % +LP
%%                    \arabic{page}                               % -LP
%                    \hfill \,}}
%%%% END PAGE 2--9 NUMBERING

%%%% PAGE 1 NUMBERING:
\makeatletter
\let\oldps@plain\ps@plain
\renewcommand{\ps@plain}{\oldps@plain%
\renewcommand{\@evenfoot}{\hspace*{-\footpageshift}\hfil %
    p.~\arabic{page} of \protect\pageref*{LastPage} % +LP
%    p.~\arabic{page}                               % -LP
    \hfil}%
\renewcommand{\@oddfoot}{\@evenfoot}}
\makeatother
%%%% END PAGE 1 NUMBERING

% Finally, give us PDF bookmarks and colored links
%
% NOTE: Some OCR software might be negatively affected by hyperlinks. So
%       most employers recommend the draft option here. Alternatively,
%       making all links black (as opposed to darkblue) should hopefully
%       prevent problems with most OCR.
%
% (to enable hyperlinks and bookmarks, comment out ``draft'' line;
%  to disable hyperlinks and bookmarks, uncomment ``draft'' line)
\usepackage{color,hyperref}
\definecolor{darkblue}{rgb}{0.0,0.0,0.3}
\hypersetup{breaklinks,colorlinks,
            linkcolor=black,urlcolor=black,
            anchorcolor=black,citecolor=black,
            %linkcolor=darkblue,urlcolor=darkblue,
            %anchorcolor=darkblue,citecolor=darkblue,
            %draft
            }

%%%%%%%%%%%%%%%%%%%%%%%% End Document Setup %%%%%%%%%%%%%%%%%%%%%%%%%%%%


%%%%%%%%%%%%%%%%%%%%%%%%%%% Helper Commands %%%%%%%%%%%%%%%%%%%%%%%%%%%%

%%% HEADING AT TOP OF CURRICULUM VITAE

% The title (name) with a horizontal rule under it
% (optional argument typesets an object right-justified across from name
%  as well)
%
% Usage: \makeheading{name}
%        OR
%        \makeheading[right_object]{name}
%
% Place at top of document. It should be the first thing.
% If ``right_object'' is provided in the square-braced optional
% argument, it will be right justified on the same line as ``name'' at
% the top of the CV. For example:
%
%       \makeheading[\emph{Curriculum vitae}]{Your Name}
%
% will put an emphasized ``Curriculum vitae'' at the top of the document
% as a title. Likewise, a picture could be included:
%
%   \makeheading[{\includegraphics[height=1.5in]{my_picture}}]{Your Name}
%
% the picture will be flush right across from the name. For this example
% to work, make sure the extra set of curly braces is included. Also
% makes ure that \usepackage{graphicx} is somewhere in the preamble.
\newcommand{\makeheading}[2][]%
        {\hspace*{-\marginparsep minus \marginparwidth}%
         \begin{minipage}[t]{\textwidth+\marginparwidth+\marginparsep}%
             {\large \bfseries #2 \hfill #1}\\[-0.15\baselineskip]%
                 \rule{\columnwidth}{1pt}%
         \end{minipage}}

%%% SECTION HEADINGS

% The section headings. Flush left in small caps down pseudo-margin.
%
% Usage: \section{section name}
\renewcommand{\section}[1]{\pagebreak[3]%
    \vspace{1.3\baselineskip}%
    \phantomsection\addcontentsline{toc}{section}{#1}%
    \noindent\llap{\scshape\smash{\parbox[t]{\marginparwidth}{\hyphenpenalty=10000\raggedright #1}}}%
    \vspace{-\baselineskip}\par}

%%% LISTS

% This macro alters a list by removing some of the space that follows the list
% (is used by lists below)
\newcommand*\fixendlist[1]{%
    \expandafter\let\csname preFixEndListend#1\expandafter\endcsname\csname end#1\endcsname
    \expandafter\def\csname end#1\endcsname{\csname preFixEndListend#1\endcsname\vspace{-0.6\baselineskip}}}

% These macros help ensure that items in outer-type lists do not get
% separated from the next line by a page break
% (they are used by lists below)
\let\originalItem\item
\newcommand*\fixouterlist[1]{%
    \expandafter\let\csname preFixOuterList#1\expandafter\endcsname\csname #1\endcsname
    \expandafter\def\csname #1\endcsname{\let\oldItem\item\def\item{\pagebreak[2]\oldItem}\csname preFixOuterList#1\endcsname}
    \expandafter\let\csname preFixOuterListend#1\expandafter\endcsname\csname end#1\endcsname
    \expandafter\def\csname end#1\endcsname{\let\item\oldItem\csname preFixOuterListend#1\endcsname}}
\newcommand*\fixinnerlist[1]{%
    \expandafter\let\csname preFixInnerList#1\expandafter\endcsname\csname #1\endcsname
    \expandafter\def\csname #1\endcsname{\let\oldItem\item\let\item\originalItem\csname preFixInnerList#1\endcsname}
    \expandafter\let\csname preFixInnerListend#1\expandafter\endcsname\csname end#1\endcsname
    \expandafter\def\csname end#1\endcsname{\csname preFixInnerListend#1\endcsname\let\item\oldItem}}

% An itemize-style list with lots of space between items
%
% Usage:
%   \begin{outerlist}
%       \item ...    % (or \item[] for no bullet)
%   \end{outerlist}
\newlist{outerlist}{itemize}{3}
    \setlist[outerlist]{label=\enskip\textbullet,leftmargin=*}
    \fixendlist{outerlist}
    \fixouterlist{outerlist}

% An environment IDENTICAL to outerlist that has better pre-list spacing
% when used as the first thing in a \section
%
% Usage:
%   \begin{lonelist}
%       \item ...    % (or \item[] for no bullet)
%   \end{lonelist}
\newlist{lonelist}{itemize}{3}
    \setlist[lonelist]{label=\enskip\textbullet,leftmargin=*,partopsep=0pt,topsep=0pt}
    \fixendlist{lonelist}
    \fixouterlist{lonelist}

% An itemize-style list with little space between items
%
% Usage:
%   \begin{innerlist}
%       \item ...    % (or \item[] for no bullet)
%   \end{innerlist}
\newlist{innerlist}{itemize}{3}
    \setlist[innerlist]{label=\enskip\textbullet,leftmargin=*,parsep=0pt,itemsep=0pt,topsep=0pt,partopsep=0pt}
    \fixinnerlist{innerlist}

% An environment IDENTICAL to innerlist that has better pre-list spacing
% when used as the first thing in a \section
%
% Usage:
%   \begin{loneinnerlist}
%       \item ...    % (or \item[] for no bullet)
%   \end{loneinnerlist}
\newlist{loneinnerlist}{itemize}{3}
    \setlist[loneinnerlist]{label=\enskip\textbullet,leftmargin=*,parsep=0pt,itemsep=0pt,topsep=0pt,partopsep=0pt}
    \fixendlist{loneinnerlist}
    \fixinnerlist{loneinnerlist}

%%% EXTRA SPACE

% To add some paragraph space between lines.
% This also tells LaTeX to preferably break a page on one of these gaps
% if there is a needed pagebreak nearby.
\newcommand{\blankline}{\quad\pagebreak[3]}
\newcommand{\halfblankline}{\quad\vspace{-0.5\baselineskip}\pagebreak[3]}

%%% FORMATTING MACROS

% Provides a linked \doi{#1} that links doi:#1 to http://dx.doi.org/#1
\usepackage{doi}
% To change the text before the DOI, adjust this command
%\renewcommand\doitext{doi:}

% Provides a linked \url{#1} that doesn't require escape characters
\usepackage{url}

% You can adjust the style \url{} uses here:
% (options are: same, rm, sf, tt; defaults to tt)
\urlstyle{same}

% For \email{ADDRESS}, links ADDRESS to the url mailto:ADDRESS
% (uncomment to typeset the e\-/mail address in typewriter font;
%  otherwise, will be typeset in the \urlstyle above)
%\DeclareUrlCommand\emaillink{\urlstyle{tt}}
\providecommand*\emaillink[1]{\nolinkurl{#1}}
\providecommand*\email[1]{\href{mailto:#1}{\emaillink{#1}}}

\providecommand\BibTeX{{B\kern-.05em{\sc i\kern-.025em b}\kern-.08em \TeX}}
\providecommand\Matlab{\textsc{Matlab}}

% Custom hyphenation rules for words that LaTeX has trouble with
\hyphenation{bio-mim-ic-ry bio-in-spi-ra-tion re-us-a-ble pro-vid-er Media-Wiki}
\hyphenation{pro-babilistic}
%%%%%%%%%%%%%%%%%%%%%%%% End Helper Commands %%%%%%%%%%%%%%%%%%%%%%%%%%%

%%%%%%%%%%%%%%%%%%%%%%%%% Begin CV Document %%%%%%%%%%%%%%%%%%%%%%%%%%%%

\begin{document}
\thispagestyle{plain}
\makeheading[\emph{Curriculum vitae}]{Michael Riis Andersen}

\section{Personal Information}

% NOTE: Mind where the & separators and \\ breaks are in the following
%       table. Table is one row made up of three parboxes. The left
%       parbox has address info, the middle parbox has a vertical bar,
%       and the right parbox has phone and electronic contact
%       information.
%
% MACROS: \rcollength is the width of the right column of the table
%             (adjust it to your liking; default is 1.85in).
%         \spacewidth is width of area between left and right boxes.
%
\newlength{\rcollength}\setlength{\rcollength}{1.85in}%
\newlength{\spacewidth}\setlength{\spacewidth}{20pt}
%
\begin{tabular}[t]{@{}p{\textwidth-\rcollength-\spacewidth}@{}p{\spacewidth}@{}p{\rcollength}}%

% Address box
%\parbox{\textwidth-\rcollength-\spacewidth}
%&
% Uncomment to add a vertical bar in middle of contact information
%{\vrule width 0.5pt}
%\parbox[m][3\baselineskip]{\spacewidth}{} &

% Non-snail-mail contact information
%\parbox{\rcollength}

\end{tabular}

%%
%% In modern CV's, it seems like ``Objective'' is frowned upon. Instead,
%% incorporate it into a well-constructed cover letter. The ``More
%% information'' can go at the end of the CV, but it should not distract
%% from the section giving references available to contact.
%%
%
% \section{Objective}
%
% Full-time position that allows for advanced research in electrical and
% computer engineering (communications, control, software, electronics,
% and sustainability), with a particular focus on complex distributed
% systems (i.e., modeling, analysis, design, and verification)
% \begin{innerlist}
%     \item For more information, see \url{http://www.tedpavlic.com/engjobsearch/}
% \end{innerlist}

\section{Academic Interests}

Machine learning, probabilistic modelling, Bayesian statistics, approximate inference, Gaussian processes,  signal processing, neuroimaging, computational science and applied mathematics.


\section{Education}

\href{http://www.dtu.dk/}{\textbf{Technical University of Denmark}},
Denmark

\begin{outerlist}


\item[] \textit{Ph.D., Machine learning and probabilistic modelling}%
        \hfill \textbf{2017}
        \begin{innerlist}
        \item Thesis title: \emph{Probabilistic models for structured sparsity}
            \end{innerlist}
\end{outerlist}

\begin{outerlist}

\item[] \textit{M.Sc., Mathematical modelling and computation}%
        \hfill \textbf{2014}
        \begin{innerlist}
        \item GPA: 11.5 (12.0 scale)
        \item Thesis title: \emph{Sparse inference using approximate message passing} (grade: 12)
        \end{innerlist}
\end{outerlist}
\halfblankline

\href{http://eng.au.dk/}{\textbf{Engineering College of Aarhus}},
Denmark
\begin{outerlist}

\item[] \textit{B.Sc. E.E., Signal and image processing}%
        \hfill \textbf{2011}
        \begin{innerlist}
        \item GPA: 11.0 (12.0 scale)
        \item Thesis title: Embedded computer vision system for people counting (grade: 12)
        \end{innerlist}

\end{outerlist}
\halfblankline

%%%%%%%%%%%%%%%%%%%%%%%%%%%%%%%%%%%%%%%%%%%%%%%%%%%%%%%%%%%%%%%%%%%%%%%%%%%%%%%%%%%%%%%%%%%%%%%%%%%%%%%%%%%%%%%%%%%%%%%%%%%%%%%%%%
%%%% Research experience
%%%%%%%%%%%%%%%%%%%%%%%%%%%%%%%%%%%%%%%%%%%%%%%%%%%%%%%%%%%%%%%%%%%%%%%%%%%%%%%%%%%%%%%%%%%%%%%%%%%%%%%%%%%%%%%%%%%%%%%%%%%%%%%%%%

\section{Research Experience}

\href{http://www.stanford.edu/}{\textbf{Aalto University}},
Espoo, Finland
\begin{outerlist}

\item[] \textit{Postdoctoral Researcher}%
            \hfill \textbf{2017 - present }
            \begin{innerlist}
                \item Part of the Probabilistic Machine Learning (PML) group working with Aki Vehtari
            \end{innerlist}

\end{outerlist}
\halfblankline

\href{http://www.stanford.edu/}{\textbf{Stanford University}},
California, USA
\begin{outerlist}

\item[] \textit{Visiting student researcher}%
            \hfill \textbf{2015 - 2016}
            \begin{innerlist}
                \item Visited Poldrack Lab and worked on a project called "Model-based dynamic functional connectivity"
            \end{innerlist}

\end{outerlist}
\halfblankline

\href{http://www.dtu.dk/}{\textbf{Technical University of Denmark}},
Denmark
\begin{outerlist}

    \item[] \textit{Postdoctoral researcher}%
            \hfill \textbf{2017}
            \begin{innerlist}
                \item Real-time EEG source localization
            \end{innerlist}


    \item[] \textit{Student programmer}%
            \hfill \textbf{2013}
            \begin{innerlist}
                \item Real-time EEG preprocessing and automatic artifact removal
            \end{innerlist}

\end{outerlist}
\halfblankline

\href{http://www.dtu.dk/}{\textbf{Engineering College of Aarhus}},
Denmark
\begin{outerlist}
	\item[]  \textit{Research assistant}%
			 \hfill \textbf{2011 - 2012}
			 \begin{innerlist}
			 	\item Design, implementation and test of real-time computer vision algorithms
			 \end{innerlist}
			 
\end{outerlist}
\halfblankline
\vspace{0.1in}

%%%%%%%%%%%%%%%%%%%%%%%%%%%%%%%%%%%%%%%%%%%%%%%%%%%%%%%%%%%%%%%%%%%%%%%%%%%%%%%%%%%%%%%%%%%%%%%%%%%%%%%%%%%%%%%%%%%%%%%%%%%%%%%%%%
%%%% Teaching experience
%%%%%%%%%%%%%%%%%%%%%%%%%%%%%%%%%%%%%%%%%%%%%%%%%%%%%%%%%%%%%%%%%%%%%%%%%%%%%%%%%%%%%%%%%%%%%%%%%%%%%%%%%%%%%%%%%%%%%%%%%%%%%%%%%%

\section{Teaching Experience}

\href{http://www.dtu.dk/}{\textbf{Aalto University}},
Finland
\begin{outerlist}

\item[] \textit{Course responsible}%
            \hfill \textbf{2018}
            \begin{innerlist}
                \item Course name: ''Gaussian Processes - Theory and Applications''
                \item Designing course content, creating exercise materials, and  lecturing
            \end{innerlist}

\item[] \textit{Teaching assistant}%
            \hfill \textbf{2017}
            \begin{innerlist}
                \item TA in course ''Bayesian Data Analysis''
            \end{innerlist}



\item[] \textit{Supervision}%
            \hfill \textbf{2017 - present}
            \begin{innerlist}
                \item Co-supervision of visiting student researchers
                \item Co-supervision of summer interns
                \item Co-supervision of master's students
                \item Co-supervision of Ph.D. students
            \end{innerlist}





\end{outerlist}
\halfblankline


\href{http://www.dtu.dk/}{\textbf{Technical University of Denmark}},
Denmark
\begin{outerlist}

\item[] \textit{Supervsion}%
            \hfill \textbf{2014 - 2017}
                    \begin{innerlist}
            \item Co-supervised master's thesis project
            \item Co-supervised project groups in the course ''Advanced machine learning''
            \item Supervised project groups in the course ''Mathematics 1''
        \end{innerlist}

% \item[] \textit{Master thesis project co-supervisor}%
%             \hfill \textbf{2016 - 2017}
%             \begin{innerlist}
%                 \item Co-supervised a project called "Real-time EEG analysis on a smart-phone"
%             \end{innerlist}

% \item[] \textit{Project co-supervisor}%
%             \hfill \textbf{2016}
%             \begin{innerlist}
%                 \item Co-supervised 3 groups of master's students doing projects as a part of the course "Advanced machine learning"
%             \end{innerlist}

% \item[] \textit{Project supervisor}%
%             \hfill \textbf{2015}
%             \begin{innerlist}
%                 \item Supervised 5 groups of bachelor students doing projects as a part of a linear algebra \& calculus course
%             \end{innerlist}

\item[] \textit{Teaching assistant}%
            \hfill \textbf{2014}
            \begin{innerlist}
                \item TA in the course ''Non-linear signal processing''
            \end{innerlist}

    \item[] \textit{Teaching assistant}%
            \hfill \textbf{2012 - 2013}
            \begin{innerlist}
                \item TA in the course ''Mathematics 1''
            \end{innerlist}
\end{outerlist}
\halfblankline

\href{http://www.dtu.dk/}{\textbf{Engineering College of Aarhus}},
Denmark
\begin{outerlist}
    \item[]  \textit{Teaching assistant}%
             \hfill \textbf{2010 - 2011}
             \begin{innerlist}
                \item Individual teaching and guidance for students with learning difficulties
                \item Subjects: signal processing, physics, programming
             \end{innerlist}
\end{outerlist}
\halfblankline
\vspace{0.1in}

%%%%%%%%%%%%%%%%%%%%%%%%%%%%%%%%%%%%%%%%%%%%%%%%%%%%%%%%%%%%%%%%%%%%%%%%%%%%%%%%%%%%%%%%%%%%%%%%%%%%%%%%%%%%%%%%%%%%%%%%%%%%%%%%%%
%%%% Other experience
%%%%%%%%%%%%%%%%%%%%%%%%%%%%%%%%%%%%%%%%%%%%%%%%%%%%%%%%%%%%%%%%%%%%%%%%%%%%%%%%%%%%%%%%%%%%%%%%%%%%%%%%%%%%%%%%%%%%%%%%%%%%%%%%%%

\section{Other Experience}

\textbf{Terma A/S},
Denmark
\begin{outerlist}
    \item[] \textit{Engineering Training}
                \hfill \textbf{2010}
            \begin{innerlist}
                \item Software port \& optimization of framework for graphics rendering
                \item C++, real time applications, embedded software
            \end{innerlist}
\end{outerlist}

% Add a little space to nudge next ``Conference Publications'' marginpar
% down to make room for tall ``Submitted Conference Publications''
% marginpar. If there are enough submitted journal publications, this
% space will not be needed (and should be removed).
\vspace{0.1in}


% \section{Conference Talks}
% \begin{outerlist}
% \item M. R. Andersen,  O. Winther, L. K. Hansen, "Spatio-temporal spike and slab priors for MMV problems", extended abstract, Signal Processing with Adaptive Sparse Structured Representations (SPARS) 2015 (Cambrigde, UK), July 2015
% \item M. R. Andersen, A. Vehtari, O. Winther, L. K. Hansen, "Bayesian Inference for spatio-temporal spike and slab priors", Workshop on Gaussian Process Approximations (Copenhagen, Denmark), 21st of May 2015
% \item M. R. Andersen, S. T. Hansen, L. K. Hansen, "Learning The Solution Sparsity Of An Ill-Posed Linear Inverse Problem With The Variational Garrote", Machine learning for signal processing 2013 (Southampton, UK), 23rd of September 2013
% \end{outerlist}
% \halfblankline

% \section{Conference Poster Presentations}
% \begin{outerlist}
% \item M. R. Andersen, O. Koyejo, R. Poldrack, ''Model-based dynamic resting state functional connectivity'', Organization for Human Brain Mapping 2016 (Geneva, Switzerland), 29th of June 2016
% \item M. R. Andersen, O. Winther, L. K. Hansen, ''Bayesian inference for structured spike and slab priors'', Advances in Neural Information Processing Systems 2014 (Montreal, Canada), 10th of December 2014
% \end{outerlist}
% \halfblankline

\section{Peer-reviewed Conference Publications}



\begin{bibenum}

\item E. Siivola, A. Vehtari, J. Gonz\'alez, J. Vanhatalo, \textbf{M. R. Andersen}. Correcting boundary over-exploration deficiencies in bayesian optimization with virtual derivative sign observations,  \textit{Machine Learning for Signal Processing, 2018}

\item \textbf{M. R. Andersen}, O. Winther, L. K. Hansen, R. Poldrack, O. Koyejo. Bayesian Structure Learning for Dynamic Brain Connectivity, \textit{The 21st International Conference on 
Artificial Intelligence and Statistics, 2018}

\item \textbf{M. R. Andersen}, A. Vehtari, O. Winther, L. K. Hansen. Bayesian inference for spatio-temporal spike-and-slab priors, \textit{Journal of Machine Learning Research, 2018}

\item X. Meng, S. Wu, \textbf{M. R. Andersen}, J. Zhu, Z. Ni. Efficient Recovery of Structured Sparse Signals via Approximate Message Passing with Structured Spike and Slab Prior, \textit{China Communications, 2018}


\item R. S. Andersen, A. U. Eliasen, N. Pedersen, \textbf{M. R. Andersen}, S. T. Hansen, L. K. Hansen. EEG source imaging assists decoding in a face recognition task, \textit{IEEE International Conference on Acoustics, Speech and Signal Processing, 2017}

\item \textbf{M. R. Andersen},  O. Winther, L. K. Hansen. Spatio-temporal spike and slab priors for MMV problems, \textit{Signal Processing with Adaptive Sparse Structured Representations, 2015}

\item \textbf{M. R. Andersen}, O. Winther, L. K. Hansen. Bayesian inference for structured spike and slab priors, \textit{Advances in Neural Information Processing Systems}, 2014

\item \textbf{M. R. Andersen}, S. T. Hansen, L. K. Hansen. Learning the solution sparsity of an ill-posed linear inverse problem with the variational garrote, \textit{Machine Learning for Signal Processing, 2013}

\item T. Gregersen, T. Jensen, \textbf{M. R. Andersen}, L. Mortensen, J. Maselyne, E. Hessel, P. Ahrendt. Consumer Grade range cameras for monitoring pig feeding behaviour, \textit{European conference on precision livestock farming, 2013}

\item T. Gregersen, T. Jensen, \textbf{M. R. Andersen}, L. Mortensen, J. Maselyne, E. Hessel. Computer vision based monitoring of performance of an RFID based eating registration system,  \textit{Construction, engineering and environment in livestock, 2013}
\end{bibenum}
\halfblankline

\section{Peer-reviewed Workshop \& Symposium Contributions}

\begin{bibenum}
\item \textbf{M. R. Andersen}, E. Siivola, G. Riutort-Mayol,  A. Vehtari. A non-parametric probabilistic model for monotonic functions, \textit{NIPS workshop on Bayesian Non-parametrics, 2018}

\item \textbf{M. R. Andersen}, E. Siivola, A. Vehtari. Bayesian optimization of Unimodal functions, \textit{NIPS workshop on Bayesian Optimization, 2017}

\item T. Paananen, J. Piironen, \textbf{M. R. Andersen,} A Vehtari. Model selection for Gaussian processes utilizing sensitivity of posterior predictive distribution, \textit{NIPS Symposium on Interpretable machine learning, 2017}
\end{bibenum}
\halfblankline

\section{Technical Reports}

\begin{bibenum}
	\item \textbf{M. R. Andersen}, T. Jensen, P. Lisouski, A.K. Mortensen, M.K. Hansen, T. Gregersen, P. Ahrendt, Kinect depth sensor evaluation for computer vision algorithms, \textit{Aarhus University, Department of Engineering, 2012}
\end{bibenum}
\halfblankline

\section{Papers Under Peer-review}




\begin{bibenum}
    \item G. Riutort-Mayol, \textbf{M. R. Andersen}, A. Vehtari,  J. L. Lerma. Spatio-temporal Gaussian process with derivative information for modelling MFS measurements, \textit{Submitted to Biometrics, 2018}

    \item \textbf{M. R. Andersen}, E. Siivola, G. Riutort-Mayol,  A. Vehtari. A non-parametric probabilistic model for monotonic functions, \textit{Submitted to AISTATS, 2018}

    \item T. Paananen, J. Piironen, \textbf{M. R. Andersen}, A. Vehtari. Variable selection for Gaussian processes via sensitivity analysis of the posterior predictive distribution, \textit{Submitted to AISTATS, 2018}

    \item W. J. Wilkinson, \textbf{M. R. Andersen}, J. D. Reiss, D. Stowell, A. Solin. Unifying probabilistic models for time-frequency analysis, \textit{Submitted to ICASSP, 2018}
\end{bibenum}
\halfblankline


% \section{Technical Reports}

% \begin{bibenum}
%     \item \textbf{M. R. Andersen}, T. Jensen, P. Lisouski, A.K. Mortensen, M.K. Hansen, T. Gregersen, P. Ahrendt, Kinect depth sensor evaluation for computer vision algorithms, \textit{Aarhus University, Department of Engineering, 2012}

% \end{bibenum}
% \halfblankline


% \section{In Preparation}
% \begin{bibenum}
    % \item M. R. Andersen, O. Koyejo, O. Winther, L. K. Hansen, R. Poldrack, "Model-based time-varying covariance estimation"

% \end{bibenum}



\end{document}

%%%%%%%%%%%%%%%%%%%%%%%%%% End CV Document %%%%%%%%%%%%%%%%%%%%%%%%%%%%%

%----------------------------------------------------------------------%
% The following is copyright and licensing information for
% redistribution of this LaTeX source code; it also includes a liability
% statement. If this source code is not being redistributed to others,
% it may be omitted. It has no effect on the function of the above code.
%----------------------------------------------------------------------%
% Copyright (c) 2007, 2008, 2009, 2010, 2011 by Theodore P. Pavlic
%
% Unless otherwise expressly stated, this work is licensed under the
% Creative Commons Attribution-Noncommercial 3.0 United States License. To
% view a copy of this license, visit
% http://creativecommons.org/licenses/by-nc/3.0/us/ or send a letter to
% Creative Commons, 171 Second Street, Suite 300, San Francisco,
% California, 94105, USA.
%
% THE SOFTWARE IS PROVIDED "AS IS", WITHOUT WARRANTY OF ANY KIND, EXPRESS
% OR IMPLIED, INCLUDING BUT NOT LIMITED TO THE WARRANTIES OF
% MERCHANTABILITY, FITNESS FOR A PARTICULAR PURPOSE AND NONINFRINGEMENT.
% IN NO EVENT SHALL THE AUTHORS OR COPYRIGHT HOLDERS BE LIABLE FOR ANY
% CLAIM, DAMAGES OR OTHER LIABILITY, WHETHER IN AN ACTION OF CONTRACT,
% TORT OR OTHERWISE, ARISING FROM, OUT OF OR IN CONNECTION WITH THE
% SOFTWARE OR THE USE OR OTHER DEALINGS IN THE SOFTWARE.
%----------------------------------------------------------------------%